\section{SpyGlass packets}

As SpyGlass itself is just an application to display something, it obviously needs a source of data where it gets the information
from what can or must be displayed. This information comes from the so called SpyGlass packets, which are syntactically
well-defined data packets. That means, that they must be formatted according to one of the five packet types described
in chapter \ref{packet_types} ff.

All SpyGlass packets consist of a header and a payload, where the header format is equal for all packet types. A header
must always have a length of exactly 19 bytes and contain the information shown in table \ref{packet_header}. Note, that
the numbering of the bytes of a packet starts with 0.

\begin{table}[htdp]
  \begin{center}
    \begin{tabular}{l|l|l|p{3cm}}
      \textbf{Byte no.} & \textbf{type} & \textbf{content} & \textbf{comment} \\
      \hline
      \hline
      0,1 & uint16 & packet length (header + payload) & \\
      \hline
      2 & uint8 & version & currently fixed to ``2'' \\
      \hline
      3 & uint8 & syntax type &  0 = std \newline
				1 = uint8List \newline
				2 = uInt16List \newline
				3 = Int16List \newline
				4 = uInt32List \newline
				5 = Int64List \newline
				6 = FloatList \newline
				7 = variable \\
      \hline
      4 & uint8 & semantic type & \\
      \hline
      5, 6 & uint16 & sender ID & \\
      \hline
      7, 8, 9, 10 & uint32 & current time in seconds &  \\
      \hline
      11, 12 & uint16 & milliseconds of current time & \\
      \hline
      13, 14 & int16 & x-coordinate of senders position & \\
      \hline
      15, 16 & int16 & y-coordinate of senders position & \\
      \hline
      17, 18 & int16 & z-coordinate of senders position & \\
    \end{tabular}
    \caption{SpyGlass packet header}
    \label{packet_header}
  \end{center}
\end{table}



\subsection{Packet types}
\label{packet_types}

\subsection{Neighborhood-Packet}
\label{subsection:neighborhood_packet}

The payload of a Neighborhood-Packet contains a list of uint16 values which represents a list of node IDs.
Its syntax type is consequently ``uint16List''. The list must neither contain a node ID twice, nor the node
ID of the sender itself. See an overview on the payload in table \ref{neighborhood_packet_payload}.

\begin{table}[htdp]
  \begin{center}
    \begin{tabular}{l|l|l}
      \textbf{Byte no.} & \textbf{content} & \textbf{type} \\
      \hline
      \hline
      19, 20 & node ID & uint16 \\
      \hline
      20, 21 & node ID & uint16 \\
      \hline
      22, 23 & node ID & uint 16 \\
      \hline
      ... & ... & ... \\
    \end{tabular}
    \caption{Payload of a Neighborhood-Packet}
    \label{neighborhood_packet_payload}
  \end{center}
\end{table}

\subsection{Coordinates-List-Packet-2}
\label{subsection:coordinates-list-packet-2}

A Coordinates-List-Packet-2 contains a payload, which consists of a list of 2-dimensional coordinates. Each coordinate is
represented by an int16 value. Thus a 2-dimensional position needs two values and consequently the payload must have a length
of a multiple of four bytes. Obviously, its syntax type is ``int16List''. A tabular overview is given in table
\ref{coordinates_list_packet_2_payload}.

\begin{table}[htdp]
  \begin{center}
    \begin{tabular}{l|l|l}
      \textbf{Byte no.} & \textbf{content} & \textbf{type} \\
      \hline
      \hline
      19, 20 & x-coordinate of position 1 & int16 \\
      \hline
      21, 22 & y-coordinate of position 1 & int16 \\
      \hline
      23, 24 & x-coordinate of position 2 & int16 \\
      \hline
      25, 26 & y-coordinate of position 2 & int16 \\
      \hline
      27, 28 & x-coordinate of position 3 & int16 \\
      \hline
      29, 30 & y-coordinate of position 3 & int16 \\
      \hline
      ... & ... & ... \\
    \end{tabular}
    \caption{Payload of a Coordinates-List-Packet-2-Packet}
    \label{coordinates_list_packet_2_payload}
  \end{center}
\end{table}


\subsection{Coordinates-List-Packet-3}
\label{subsection:coordinates-list-packet-3}

A Coordinates-List-Packet-3 is very similar to a Coordinates-List-2-Packet. The only difference is the dimensionality, because
this packet contains positions with three dimensions instead of 2. As each coordinate is represented by an int16 value and
a 3-dimensional position needs three values, the payload must have a length of a multiple of six bytes. The syntax type is
``int16List''. See table \ref{coordinates_list_packet_3_payload} for further details.

\begin{table}[htdp]
  \begin{center}
    \begin{tabular}{l|l|l}
      \textbf{Byte no.} & \textbf{content} & \textbf{type} \\
      \hline
      \hline
      19, 20 & x-coordinate of position 1 & int16 \\
      \hline
      21, 22 & y-coordinate of position 1 & int16 \\
      \hline
      23, 24 & z-coordinate of position 1 & int16 \\
      \hline
      25, 26 & x-coordinate of position 2 & int16 \\
      \hline
      27, 28 & y-coordinate of position 2 & int16 \\
      \hline
      29, 30 & z-coordinate of position 2 & int16 \\
      \hline
      ... & ... & ... \\
    \end{tabular}
  \end{center}
  \caption{Payload of a Coordinates-List-Packet-3-Packet}
  \label{coordinates_list_packet_3_payload}
\end{table}

\subsection{Trajectory-Packet-2}
\label{subsection:trajectory2d}

The payload of Trajectory-Packet-2 represents the way of an object via several 2-dimensional positions and the time it needs
to come from one position to the next. So the payload contains a list of tuples, where the first two elements of each tuple
represents a position. The third element represents the time in seconds, that the object spends between the position given
before and the next position. Although the syntax type is ``int16List'' and all elements are of type int16, the time
element must not contain negative values.

Since the last position specifies the ending point, the last tuple is incomplete. It contains only the two elements
representing the position. Table \ref{trajectory_packet_2_payload} shows the content of the payload.

\begin{table}[htdp]
 \begin{center}
    \begin{tabular}{l|l|l}
      \textbf{Byte no.} & \textbf{content} & \textbf{type} \\
      \hline
      \hline
      19, 20 & x-coordinate of position 1 & int16 \\
      \hline
      21, 22 & y-coordinate of position 1 & int16 \\
      \hline
      23, 24 & duration of moving between & int16\\
      & position 1 and 2 & \\
      \hline
      25, 26 & x-coordinate of position 2 & int16 \\
      \hline
      27, 28 & y-coordinate of position 2 & int16 \\
      \hline
      29, 30 & duration of moving between & int16 \\
      &  position 2 and 3 & \\
      \hline
      ... & ... & ... \\
      \hline
      19 + 6(n-1), 20 + 6(n-1) & x-coordinate of position n & int16 \\
      \hline
      21 + 6(n-1), 22 + 6(n-1) & y-coordinate of position n & int16 \\
      \hline
    \end{tabular}
    \caption{Payload of a Trajectory-Packet-2-Packet}
    \label{trajectory_packet_2_payload}
  \end{center}
\end{table}

\subsection{Trajectory-Packet-3}
\label{subsection:trajectory3d}

The payload of a Trajectory-Packet-3 is very similar to a Trajectory-Packet-2. The only difference is, that the positions
are not given in 2D but in 3D. Thus each position consists of three coordinates. Between two positions the payload contains
the duration, the objects needs to pass the way from one position to the next.

The syntax type is ``int16List''. The duration value must be given in seconds and must not contain a negative number
even though it is an int16 field. See table \ref{trajectory_packet_3_payload} for more details.

\begin{table}[htdp]
 \begin{center}
    \begin{tabular}{l|l|l}
      \textbf{Byte no.} & \textbf{content} & \textbf{type} \\
      \hline
      \hline
      19, 20 & x-coordinate of position 1 & int16 \\
      \hline
      21, 22 & y-coordinate of position 1 & int16 \\
      \hline
      23, 24 & z-coordinate of position 1 & int16 \\
      \hline
      25, 26 & duration of moving between & int16\\
      & position 1 and 2 & \\
      \hline
      27, 28 & x-coordinate of position 2 & int16 \\
      \hline
      29, 30 & y-coordinate of position 2 & int16 \\
      \hline
      31, 32 & z-coordinate of position 2 & int16 \\
      \hline
      33, 34 & duration of moving between & int16 \\
      &  position 2 and 3 & \\
      \hline
      ... & ... & ... \\
      \hline
      19 + 8(n-1), 20 + 8(n-1) & x-coordinate of position n & int16 \\
      \hline
      21 + 8(n-1), 22 + 8(n-1) & y-coordinate of position n & int16 \\
      \hline
      23 + 8(n-1), 24 + 8(n-1) & z-coordinate of position n & int16 \\
      \hline
    \end{tabular}
    \caption{Payload of a Trajectory-Packet-3-Packet}
    \label{trajectory_packet_3_payload}
  \end{center}
\end{table}

\subsection{Data-Packet}
\label{subsection:datapacket}

A Data-Packets payload contains arbitrary data. Its syntax type is flexible but must be defined in the header of the packet.
The possible syntax types are ``uint8List'', ``uint16List'', int16List``, ''uint32List``, ''int64List`` and ''floatList``.
The length of the payload depends on the syntax type.
